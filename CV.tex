%%%%%%%%%%%%%%%%%%%%%%%%%%%%%%%%%%%%%%%%%
% Plasmati Graduate CV
% LaTeX Template
% Version 1.0 (24/3/13)
%
% This template has been downloaded from:
% http://www.LaTeXTemplates.com
%
% Original author:
% Alessandro Plasmati (alessandro.plasmati@gmail.com)
%
% License:
% CC BY-NC-SA 3.0 (http://creativecommons.org/licenses/by-nc-sa/3.0/)
%
% Important note:
% This template needs to be compiled with XeLaTeX.
% The main document font is called Fontin and can be downloaded for free
% from here: http://www.exljbris.com/fontin.html
%
%%%%%%%%%%%%%%%%%%%%%%%%%%%%%%%%%%%%%%%%%

%----------------------------------------------------------------------------------------
%	PACKAGES AND OTHER DOCUMENT CONFIGURATIONS
%----------------------------------------------------------------------------------------

\documentclass[a4paper,10pt]{article} % Default font size and paper size

\usepackage{textcomp}

\usepackage[T1]{fontenc}
\renewcommand{\familydefault}{\sfdefault}

% \usepackage{fontspec} % For loading fonts
% \defaultfontfeatures{Mapping=tex-text}
% \setmainfont[SmallCapsFont = Fontin SmallCaps]{Fontin} % Main document font

\usepackage{xunicode,xltxtra,url,parskip} % Formatting packages

\usepackage[usenames,dvipsnames]{xcolor} % Required for specifying custom colors
\definecolor{azureblue}{rgb}{0.0, 0.5, 1.0}
\definecolor{battleshipgrey}{rgb}{0.47, 0.5, 0.8}
\usepackage{fullpage}
% \usepackage[big]{layaureo} % Margin formatting of the A4 page, an alternative to layaureo can be \usepackage{fullpage}
% To reduce the height of the top margin uncomment: \addtolength{\voffset}{-1.3cm}

\usepackage{hyperref} % Required for adding links	and customizing them
\definecolor{linkcolour}{rgb}{0,0.2,0.6} % Link color
\hypersetup{colorlinks,breaklinks,urlcolor=linkcolour,linkcolor=linkcolour} % Set link colors throughout the document



\usepackage[a4paper, top=2cm, bottom=2cm, left=2.2cm, right=2.2cm]{geometry}
\usepackage{titlesec} % Used to customize the \section command
\titleformat{\section}{\bfseries\large\raggedright}{}{0em}{}[\titlerule] % Text formatting of sections
\titlespacing{\section}{0pt}{10pt}{2pt} % Spacing around sections

\setlength{\titlewidth}{\textwidth}

\begin{document}

\pagestyle{empty} % Removes page numbering



%----------------------------------------------------------------------------------------
%	NAME AND CONTACT INFORMATION
%----------------------------------------------------------------------------------------

\par{\centering{\Huge \textsc{Pedro Valero}}\bigskip\par} % Your name

\section{Personal Data}

\begin{tabular}{rl}
\textsc{Phone:} & +44 7741 767 126\\
\textsc{email:} & \href{mailto:pedro.valero.mejia@gmail.com}{pedro.valero.mejia@gmail.com} \\
\textsc{webpage:} & \href{https://pevalme.github.io/}{https://pevalme.github.io/}\\
\end{tabular}

\section{Work Experience}

\begin{tabular}{p{2.5cm}p{13.2cm}}
{\small September 2020} & \textbf{Facebook} (London, United Kingdom) \\
{\small Current} & \textcolor{azureblue}{\textbf{Product Engineer}} \hfill  \\
& We develop, analyze and maintain the infrastructure required for ads that allow users to open a conversation on WhatsApp.
As a Product Engineer, my work involves full stack development, carring out the required changes in the backend and the UI for each of the features that I develop.
\end{tabular}

\begin{tabular}{p{2.5cm}p{13.2cm}}
{\small July 2019} & \textbf{Facebook} (Palo Alto, California) \\
{\small October 2019} & \textcolor{azureblue}{\textbf{Research Intern at the Data Compression Team}} \hfill Manager: \emph{Yann Collet} \\
& We analyzed the strengths of grammar-based compression to understand how it could be used within the next generation of data compressors.
In order to perform this analysis, I built a prototype (in \emph{C}) of a grammar-based compressor that achieved compression ratios comparable to the ones obtained with \href{https://facebook.github.io/zstd/}{zstd}.
\end{tabular}

\begin{tabular}{p{2.5cm}p{13.2cm}}
{\small September 2016} & \textbf{IMDEA Software Institute} (Madrid, Spain) \\
{\small September 2020} & \textcolor{azureblue}{\textbf{PhD Student}}\hfill Manager: \emph{Pierre Ganty}\\

& My PhD is focused on Applications of Language Theory.
Some of the most relevant projects I have worked on as part of my PhD are:\\
[3pt]
& \textcolor{battleshipgrey}{\textbf{Searching on Compressed Text}} \\
% [2pt]
& We studied the problem of searching with regular expressions on compressed text without decompression. We devised a simple technique that speeds up the search by taking advantage of the information about repetitions on the text extracted by the compressor.
We implemented a tool, \emph{zearch}, for counting the number of lines in a compressed file that match a given regular expression.
The tool outperforms the state of the art decompress-and-search approach.
The results were published at the \emph{Data Compression Conference}. \\
[3pt]
& \textcolor{battleshipgrey}{\textbf{Automata Minimization}} \\
% [2pt]
& We defined a framework of automata constructions based on equivalences over words that unifies the different existing techniques for automata minimization.
The most relevant aspect of this work is that our framework covers the Brzozowski's algorithm, which was previously considered as orthogonal to the rest of minimization algorithms.
This work was published at the \emph{Mathematical Foundations of Computer Science} conference.
\end{tabular}

\begin{tabular}{p{2.5cm}p{13.2cm}}
{\small September 2015} & \textbf{IMDEA Software Institute} (Madrid, Spain) \\
{\small May 2016} & \textcolor{azureblue}{\textbf{Part-time Intern}} \hfill Manager: \emph{Pierre Ganty} \\
& During this internship, we studied whether different idioms, common among network protocols, could be validated with parser generators for context-free languages.
We implemented (with \emph{flex} and \emph{bison}) a modular, robust, and efficient input validator for HTTP relying on context-free grammars and regular expressions.
We published the obtained results at the \emph{Automated Technology for Verification and Analysis} conference.
\end{tabular}

\begin{tabular}{p{2.5cm}p{13.2cm}}
{\small June 2015} & \textbf{Max Planck Institute for Software Systems} (Kaiserslautern, Germany) \\
{\small September 2015} & \textcolor{azureblue}{\textbf{Intern}} \hfill Manager: \emph{Rupak Majumdar} \\
& We used the \href{https://www.ros.org/}{Robot Operative System} to simulate a robot which was controlled by a combination of voice commands and hand gestures (captured with a \href{https://www.leapmotion.com/}{Leap Motion} device).
I designed and implemented a system to handle the given commands and execute them according to their predefined priorities.
\end{tabular}

\begin{tabular}{p{2.5cm}p{13.2cm}}
{\small June 2014} & \textbf{IMDEA Software Institute} (Madrid, Spain) \\
{\small May 2015} & \textcolor{azureblue}{\textbf{Intern}} \hfill Manager: \emph{Pierre Ganty}\\
& The goal of this project was to update the tool \href{https://github.com/pierreganty/mist/wiki}{mist}, a safety checker for Petri Nets and extensions developed by my supervisor, Pierre Ganty.
In particular, I implemented new \emph{Python} scripts to better test and benchmark the tool and improved the presentation of the results by using the \emph{JavaScript} library \href{https://d3js.org/}{\emph{D3}}.
\end{tabular}

\section{Software}

\begin{tabular}{p{2.5cm}p{13.2cm}}
{\small HTTValidator} & An input validator for HTTP messages that relies on recognizers for context-free and regular languages (implemented using Bison and Flex respectively) to perform the validation. \emph{Publicly available on \href{https://github.com/pevalme/HTTPValidator}{GitHub}}. \\

{\small Zearch} & A tool for regular expression searching on grammar-compressed text (implemented in C). Publicly available on \emph{\href{https://github.com/pevalme/zearch}{GitHub}}.\\
\end{tabular}

\section{Programming Skills}
\begin{tabular}{p{2.5cm}p{13.2cm}}
\small{Languages} & \textcolor{azureblue}{\textbf{Advanced:}} C, Hack, React, Python, \LaTeX.\\
[1pt]
& \textcolor{battleshipgrey}{\textbf{Medium}}: C++, Java, Bash, Awk, JavaScript, PHP, HTML, CSS.\\
& \textbf{Basic}: R, SQL, Assembly, Lisp, Prolog.\\
[3pt]
\small{Software} & Linux, Sublime Text, Atom, Git, svn, mercurial, Zsh.\\
\end{tabular}

\section{Publications}
\begin{tabular}{p{2.5cm}p{13.2cm}}
\small{Fundamenta} & \textsc{A Congruence-Based Perspective on Finite Tree Automata} \\
\small{Informaticae 2021} & \textit{with Elena Gutiérrez and Pierre Ganty}. \\
[3pt]
\small{TOCL 2021} & \textsc{Complete Abstractions for Checking Language Inclusion} \\
& \textit{with Francesco Ranzato and Pierre Ganty}. \\
[3pt]
\small{MFCS 2020} & \textsc{A Quasiorder-based Perspective on Residual Automata} \\
& \textit{with Elena Gutiérrez and Pierre Ganty}. \\
[3pt]
\small{SAS 2019} & \textsc{Complete Abstractions for Checking Language Inclusion} \\
& \textit{with Francesco Ranzato and Pierre Ganty}. \\
[3pt]
\small{MFCS 2019} & \textsc{A Congruence-based Perspective on Automata Minimization Algorithms} \\
 & \textit{with Elena Gutiérrez and Pierre Ganty}. \\
[3pt]
\small{DCC 2019} & \textsc{Regular Expression Searching on Compressed Text} \\
 & \textit{with Pierre Ganty}. \\
[3pt]
\small{ATVA 2017} & \textsc{A Language-Theoretic View on Network Protocols} \\
& \textit{with Pierre Ganty and Boris Köpf}. \\
\end{tabular}

\section{Committees}
As a PhD student I have contributed to the organization of the \href{http://atva2019.iis.sinica.edu.tw/organization/}{ATVA'19} and \href{https://conf.researchr.org/track/etaps-2019/tacas-2019-papers#Artifact-Evaluation}{TACAS'19} conferences as a member of the \emph{Artifact Evaluation Committee}.
The goal of these committees is to check consistency and replicability of results presented in submitted papers as well as evaluating their completeness, documentation and ease of use.

\section{Education}
\begin{tabular}{p{2.5cm}p{13.2cm}}
\small{2016 - 2020} & \textsc{PhD in Software, Systems and Computing} \\
& at \textbf{Universidad Politécnica de Madrid}\\
& Graduated \emph{Cum Laude} \\
[3pt]
\small{2011 - 2016} & \textsc{Double degree at Computer Science and Mathematics}\\
& at \textbf{Universidad Autónoma de Madrid} \\
& Obtained four consecutive \emph{Excellence Awards} for academic performance. \\
& \normalsize \textsc{gpa}: 9.14/10.0
\end{tabular}
\end{document}
