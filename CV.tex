%%%%%%%%%%%%%%%%%%%%%%%%%%%%%%%%%%%%%%%%%
% Plasmati Graduate CV
% LaTeX Template
% Version 1.0 (24/3/13)
%
% This template has been downloaded from:
% http://www.LaTeXTemplates.com
%
% Original author:
% Alessandro Plasmati (alessandro.plasmati@gmail.com)
%
% License:
% CC BY-NC-SA 3.0 (http://creativecommons.org/licenses/by-nc-sa/3.0/)
%
% Important note:
% This template needs to be compiled with XeLaTeX.
% The main document font is called Fontin and can be downloaded for free
% from here: http://www.exljbris.com/fontin.html
%
%%%%%%%%%%%%%%%%%%%%%%%%%%%%%%%%%%%%%%%%%

%----------------------------------------------------------------------------------------
%	PACKAGES AND OTHER DOCUMENT CONFIGURATIONS
%----------------------------------------------------------------------------------------

\documentclass[a4paper,10pt]{article} % Default font size and paper size

\usepackage{textcomp}

\usepackage[T1]{fontenc}
\renewcommand{\familydefault}{\sfdefault}

% \usepackage{fontspec} % For loading fonts
% \defaultfontfeatures{Mapping=tex-text}
% \setmainfont[SmallCapsFont = Fontin SmallCaps]{Fontin} % Main document font

\usepackage{xunicode,xltxtra,url,parskip} % Formatting packages

\usepackage[usenames,dvipsnames]{xcolor} % Required for specifying custom colors
\definecolor{azureblue}{rgb}{0.0, 0.5, 1.0}
\definecolor{battleshipgrey}{rgb}{0.47, 0.5, 0.8}
\usepackage{fullpage}
% \usepackage[big]{layaureo} % Margin formatting of the A4 page, an alternative to layaureo can be \usepackage{fullpage}
% To reduce the height of the top margin uncomment: \addtolength{\voffset}{-1.3cm}

\usepackage{hyperref} % Required for adding links	and customizing them
\definecolor{linkcolour}{rgb}{0,0.2,0.6} % Link color
\hypersetup{colorlinks,breaklinks,urlcolor=linkcolour,linkcolor=linkcolour} % Set link colors throughout the document



\usepackage[a4paper, top=2cm, bottom=2cm, left=2.2cm, right=2.2cm]{geometry}
\usepackage{titlesec} % Used to customize the \section command
\titleformat{\section}{\bfseries\large\raggedright}{}{0em}{}[\titlerule] % Text formatting of sections
\titlespacing{\section}{0pt}{10pt}{2pt} % Spacing around sections

\setlength{\titlewidth}{\textwidth}

\begin{document}

\pagestyle{empty} % Removes page numbering



%----------------------------------------------------------------------------------------
%	NAME AND CONTACT INFORMATION
%----------------------------------------------------------------------------------------

\par{\centering{\Huge \textsc{Pedro Valero}}\bigskip\par} % Your name

\section{Personal Data}

\begin{tabular}{rl}
\textsc{Phone:} & +44 7741 767 126\\
\textsc{email:} & \href{mailto:pedro.valero.mejia@gmail.com}{pedro.valero.mejia@gmail.com} \\
\textsc{webpage:} & \href{https://pevalme.github.io/}{https://pevalme.github.io/}\\
\end{tabular}

\section{Work Experience}

\begin{tabular}{p{2.5cm}p{13.2cm}}
{\small September 2020} & \textbf{Meta} (London, United Kingdom) \\
{\small Current} & \textcolor{azureblue}{\textbf{Software Engineer}} \hfill  \\
& Since I joined Meta, I have worked in three different teams with very different scope:
\begin{itemize}
\item \emph{Click-to-WhatsApp} (as IC), where our goal was to drive adoption and improve the performance of ads that lead users to open a conversation via WhatsApp with the advertisers.
\item \emph{Creators as Marketers} (as IC), where we developed from scratch the Affiliate program. As a backend engineer, I worked on the cross-app infrastructure to allow content creators to
discovery products to recommend, as well as allow advertisers to discover creators to partner with.
\item \emph{Catalog} (as TL), where I am currenting leading a team of 8 people to improve the sourcing and utiliation of signals relative to user behaviour to improve the performance of Dynamic Ads.\vspace*{-\baselineskip}
\end{itemize}
\end{tabular}

\begin{tabular}{p{2.5cm}p{13.2cm}}
{\small July 2019} & \textbf{Facebook} (Palo Alto, California) \\
{\small October 2019} & \textcolor{azureblue}{\textbf{Research Intern at the Data Compression Team}} \\
& Built a prototype (in \emph{C}) of a grammar-based compressor that achieved compression ratios comparable to the ones obtained with \href{https://facebook.github.io/zstd/}{zstd}.
\end{tabular}

\begin{tabular}{p{2.5cm}p{13.2cm}}
{\small September 2016} & \textbf{IMDEA Software Institute} (Madrid, Spain) \\
{\small September 2020} & \textcolor{azureblue}{\textbf{PhD Student}}\hfill PhD Advisor: \emph{Pierre Ganty}\\

& My PhD is focused on Applications of Language Theory.
The most relevant project I have worked on as part of my PhD, which let to an Internship at Meta, was the development of \emph{zearch}, a tool
for searching with regular expressions in compressed text which outperformed the state of the art technology.
The details of this work were published at the \emph{Data Compression Conference}. \\
\end{tabular}

\begin{tabular}{p{2.5cm}p{13.2cm}}
{\small September 2015} & \textbf{IMDEA Software Institute} (Madrid, Spain) \\
{\small May 2016} & \textcolor{azureblue}{\textbf{Part-time Intern}} \hfill Manager: \emph{Pierre Ganty} \\
& Analysed different network protocols and whether they could be validated with parser generators for context-free languages.
We developed a modular, robust, and efficient input validator for HTTP relying on context-free grammars and regular expressions.
\end{tabular}

\begin{tabular}{p{2.5cm}p{13.2cm}}
{\small June 2015} & \textbf{Max Planck Institute for Software Systems} (Kaiserslautern, Germany) \\
{\small September 2015} & \textcolor{azureblue}{\textbf{Intern}} \hfill Manager: \emph{Rupak Majumdar} \\
& Designed a system to control a robot using by voice commands and gestures. The system was implemented and simulated with \href{https://www.ros.org/}{Robot Operative System}.
\end{tabular}

\begin{tabular}{p{2.5cm}p{13.2cm}}
{\small June 2014} & \textbf{IMDEA Software Institute} (Madrid, Spain) \\
{\small May 2015} & \textcolor{azureblue}{\textbf{Intern}} \hfill Manager: \emph{Pierre Ganty}\\
& Improved the infrastructure for testing and benchmarking \href{https://github.com/pierreganty/mist/wiki}{mist}, a safety checker for Petri Nets and extensions.
\end{tabular}

\section{Software}

\begin{tabular}{p{2.5cm}p{13.2cm}}
{\small HTTValidator} & An input validator for HTTP messages that relies on recognizers for context-free and regular languages (implemented using Bison and Flex respectively) to perform the validation. \emph{Publicly available on \href{https://github.com/pevalme/HTTPValidator}{GitHub}}. \\

{\small Zearch} & A tool for regular expression searching on grammar-compressed text (implemented in C). Publicly available on \emph{\href{https://github.com/pevalme/zearch}{GitHub}}.\\
\end{tabular}

\section{Programming Skills}
\begin{tabular}{p{2.5cm}p{13.2cm}}
\small{Languages} & \textcolor{azureblue}{\textbf{Advanced:}} C, Hack, React, Python, SQL.\\
[1pt]
& \textcolor{battleshipgrey}{\textbf{Medium}}: C++, Java, Bash, Awk, JavaScript, PHP, HTML, CSS, \LaTeX.\\
& \textbf{Basic}: R, Assembly, Lisp, Prolog.\\
[3pt]
\small{Software} & Linux, Sublime Text, Atom, Git, svn, mercurial, Zsh.\\
\end{tabular}

\section{Publications}
\begin{tabular}{p{2.5cm}p{13.2cm}}
\small{Fundamenta} & \textsc{A Congruence-Based Perspective on Finite Tree Automata} \\
\small{Informaticae 2021} & \textit{with Elena Gutiérrez and Pierre Ganty}. \\
[3pt]
\small{TOCL 2021} & \textsc{Complete Abstractions for Checking Language Inclusion} \\
& \textit{with Francesco Ranzato and Pierre Ganty}. \\
[3pt]
\small{MFCS 2020} & \textsc{A Quasiorder-based Perspective on Residual Automata} \\
& \textit{with Elena Gutiérrez and Pierre Ganty}. \\
[3pt]
\small{SAS 2019} & \textsc{Complete Abstractions for Checking Language Inclusion} \\
& \textit{with Francesco Ranzato and Pierre Ganty}. \\
[3pt]
\small{MFCS 2019} & \textsc{A Congruence-based Perspective on Automata Minimization Algorithms} \\
 & \textit{with Elena Gutiérrez and Pierre Ganty}. \\
[3pt]
\small{DCC 2019} & \textsc{Regular Expression Searching on Compressed Text} \\
 & \textit{with Pierre Ganty}. \\
[3pt]
\small{ATVA 2017} & \textsc{A Language-Theoretic View on Network Protocols} \\
& \textit{with Pierre Ganty and Boris Köpf}. \\
\end{tabular}

\section{Committees}
As a PhD student I have contributed to the organization of the \href{http://atva2019.iis.sinica.edu.tw/organization/}{ATVA'19} and \href{https://conf.researchr.org/track/etaps-2019/tacas-2019-papers#Artifact-Evaluation}{TACAS'19} conferences as a member of the \emph{Artifact Evaluation Committee}.
The goal of these committees is to check consistency and replicability of results presented in submitted papers as well as evaluating their completeness, documentation and ease of use.

\section{Education}
\begin{tabular}{p{2.5cm}p{13.2cm}}
\small{2016 - 2020} & \textsc{PhD in Software, Systems and Computing} \\
& at \textbf{Universidad Politécnica de Madrid}\\
& Graduated \emph{Cum Laude} \\
[3pt]
\small{2011 - 2016} & \textsc{Double degree at Computer Science and Mathematics}\\
& at \textbf{Universidad Autónoma de Madrid} \\
& Obtained four consecutive \emph{Excellence Awards} for academic performance. \\
& \normalsize \textsc{gpa}: 9.14/10.0
\end{tabular}
\end{document}
